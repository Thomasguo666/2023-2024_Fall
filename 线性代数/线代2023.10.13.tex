% !TeX root = 线性代数.tex

\begin{example}
    求投影矩阵
    \begin{equation}
      A = \begin{bmatrix}
       1 & 1\\
       0 & 1\\
       0 & 0\\
      \end{bmatrix}.
    \end{equation}
    计算投影矩阵$P = A \left( A A^{\mathrm{T}} \right) ^{-1}A^{\mathrm{T}}$,
    \begin{equation}
      A^{\mathrm{T}}A = \begin{bmatrix}
       1 & 0 & 0\\
       1 & 1 & 0\\
      \end{bmatrix}
      \begin{bmatrix}
       1 & 1\\
       0 & 1\\
       0 & 0\\
      \end{bmatrix}
      =
      \begin{bmatrix}
       1 & 1\\
       1 & 2\\
      \end{bmatrix},
    \end{equation}
    \begin{equation}
      \left( A^{\mathrm{T}}A \right) ^{-1}  = \frac{1}{1\times 2-1\times 1} \begin{bmatrix}
       2 & -1\\
       -1 & 1\\
      \end{bmatrix}
      =
      \begin{bmatrix}
       2 & -1\\
       -1 & 1\\
      \end{bmatrix}.
    \end{equation}  
    于是可得
    \begin{equation}
        \begin{aligned}
            P = A \left( A^{\mathrm{T}}A \right) A^{\mathrm{T}} & = 
            \begin{bmatrix}
             1 & 1\\
             0 & 1\\
             0 & 0\\
            \end{bmatrix}
            \begin{bmatrix}
             2 & -1\\
             -1 & 1\\
            \end{bmatrix}
            \begin{bmatrix}
             1 & 0 & 0\\
             1 & 1 & 0\\
            \end{bmatrix}
            \\
            & =
            \begin{bmatrix}
             1 & 0\\
             -1 & 1\\
             0 & 0\\
            \end{bmatrix}
            \begin{bmatrix}
             1 & 0 & 0\\
             1 & 1 & 0\\
            \end{bmatrix}
            \\
            & =
            \begin{bmatrix}
             1 & 0 & 0\\
             0 & 1 & 0\\
             0 & 0 & 0\\
            \end{bmatrix}.
        \end{aligned}
    \end{equation}
\end{example}

下面我们考虑什么时候$A^{\mathrm{T}}A$可逆的问题.
\begin{proposition}
    $A^{\mathrm{T}}A$的零空间和$A$的零空间是一样的.
\end{proposition}

\begin{proof}
    如果$Ax=0$, 那么两边左乘$A^{\mathrm{T}}$, 得到
    \begin{equation}
      A^{\mathrm{T}}A x = 0.
    \end{equation}

    反之, 如果$A^{\mathrm{T}}Ax=0$, 两边左乘$x^{\mathrm{T}}$, 得到
    \begin{equation}
      x^{\mathrm{T}} A^{\mathrm{T}} A x = \left( Ax \right) ^{\mathrm{T}} Ax = 0.
    \end{equation}
    也就是说, $Ax$的模长为零, 那它必然是零向量.
\end{proof}

根据上述命题, 我们可以发现: 假设$A^{\mathrm{T}}A$可逆$\iff$ $A^{\mathrm{T}}Ax = 0$只有零解 $\iff$ $Ax =0 $只有零解.

这意味着:
\begin{itemize}
    \item $A$的列向量是线性无关的.
    
    \item $A$的秩为$r = n$. 
\end{itemize}

\subsection{正交投影的应用}
\subsubsection{最小二乘法}
在以后的学习中, 会经常遇到求以下函数的极小值
\begin{equation}
  f(x) = \left| Ax - b \right|
\end{equation}
其中的$A$是一个$m \times n$的矩阵, $b$是一个$m \times 1$的向量.
\begin{itemize}
    \item 如果$b \in C(A)$, 那么$f(x)$的极小值为$0$, $x$的解为$A x = b$的解.
    \item 如果$b \notin C(A)$, 这时候极小值的$x$对应正交投影的坐标
    \begin{equation}
      \hat{x} = \left( A^{\mathrm{T}}A \right) ^{-1} A^{\mathrm{T}} b.
    \end{equation}
\end{itemize}

\subsubsection{线性回归}
收集到一些数据, $(y_1,t_1), (y_2,t_2) , \cdots, (y_m, t_m)$. 假设$y$和$t$之间有一个线性关系$y = Dt +C$, 用数据去估计$C$和$D$.

估计方法: 考虑一个损失函数
\begin{equation}
  L = \sum_{i=1}^{m} \left( y(t_i)  - y_i \right) ^{2} = \sum_{i=1}^{m} \left( C + D t_i  - y_i \right) ^{2} = \left| Ax -b \right| ^{2}
\end{equation}
其中
\begin{equation}
  A = \begin{bmatrix}
   1 & t_1\\
   1 & t_2\\
   \vdots & \vdots\\
   1 & t_m\\
  \end{bmatrix},
  \quad 
  x = \begin{bmatrix}
   C\\
   D\\
  \end{bmatrix},
  \quad 
  b = \begin{bmatrix} 
  y_1 \\ 
  y_2 \\ 
  \vdots \\ 
  y_m 
  \end{bmatrix}.
\end{equation}
我们做一些计算\
\begin{equation}
  A^{\mathrm{T}}A = \begin{bmatrix}
   1 & 1 & \cdots & 1\\
   t_1 & t_2 & \cdots & t_m\\
  \end{bmatrix}
  \begin{bmatrix}
    1 & t_1\\
    1 & t_2\\
    \vdots & \vdots\\
    1 & t_m\\
   \end{bmatrix}
   =
   \begin{bmatrix}
    m & \sum_{i=1}^{m} t_i\\
    \sum_{i=1}^{m} t_i & \sum_{i=1}^{m} t_i^{2}\\
   \end{bmatrix},
\end{equation}
\begin{equation}
  A^{\mathrm{T}}b = \begin{bmatrix}
   \sum_{i=1}^{m} y_i\\
   \sum_{i=1}^{m} y_i t_i\\
  \end{bmatrix},
\end{equation}
\begin{equation}
  \left( A^{\mathrm{T}}A \right) ^{-1}
  =
  \frac{1}{m \sum_{i=1}^{m} t_i^{2} - \left( \sum_{i=1}^{m} t_i \right) } \begin{bmatrix}
   \sum_{i=1}^{m} t_i^{2} & -\sum_{i=1}^{m} t_i\\
   -\sum_{i=1}^{m} t_i & m\\
  \end{bmatrix}
\end{equation}
最后我们可以得到
\begin{equation}
  C = \frac{\sum_{i=1}^{m} y_i \sum_{i=1}^{m} t_i^{2} - \sum_{i=1}^{m} t_i \sum_{i=1}^{m} y_i t_i}{m \sum_{i=1}^{m} t_i^{2} - \left( \sum_{i=1}^{m} t_i \right) ^{2}}
  ,\quad
  D = \frac{m\sum_{i=1}^{m} y_i t_i - \sum_{i=1}^{m} y_i \sum_{i=1}^{m} t_i}{m \sum_{i=1}^{m} t_i^{2} - \left( \sum_{i=1}^{m} t_i \right) ^{2}}.
\end{equation}

\subsubsection{正交基}
一组基需要满足条件:
\begin{itemize}
    \item 线性无关.
    \item 任何向量都可以写成$\left( v_1,v_2,\cdots,v_m \right) $的线性组合.
\end{itemize}

\begin{definition}
    $\left( q_1,q_2,\cdots,q_n \right) $是一组基, 且满足
    \begin{equation}
      q_i \cdot q_j = \delta_{ij}.
    \end{equation}
    则它们是正交归一基.
\end{definition}

如果$q_1,q_2,\cdots,q_n$是一组正交归一基, 那么对应的矩阵$Q = \begin{bmatrix} q_1, q_2, \cdots, q_n \end{bmatrix}$满足
\begin{equation}
  Q^{\mathrm{T}}Q = Q Q^{\mathrm{T}} = I.
\end{equation}
验证:
\begin{equation}
  Q^{\mathrm{T}}Q = \begin{bmatrix} 
  q^{\mathrm{T}}_1 \\ 
  q^{\mathrm{T}}_2 \\ 
  \vdots \\ 
  q^{\mathrm{T}}_n 
  \end{bmatrix}
  \begin{bmatrix} q^{\mathrm{T}}_1, q^{\mathrm{T}}_2, \cdots, q^{\mathrm{T}}_n \end{bmatrix}
  =
  \begin{bmatrix}
    \ddots & &\\
    &q^{\mathrm{T}}_j q_i & \\
    &  &\ddots \\
  \end{bmatrix}
  =
  \begin{bmatrix}
    \ddots &  & \\
    & \delta_{ij} & \\
    &  & \ddots\\
  \end{bmatrix}
  =
  I.
\end{equation}

给定一组基, 可以构造一组正交归一基(Gram-Schmit).
方法如下:
\begin{enumerate}
    \item 选一个向量$a$, 令矩阵$A = [a]$.
    
    \item 通过$b$构造一个向量$B$, 要求$B$垂直于$A$的列向量.
    那么,
    \begin{equation}
      B  = b - A \left( A^{\mathrm{T}}A \right) ^{-1} A^{\mathrm{T}} b.
    \end{equation}

    \item 通过$c$构造一个向量$C$,
    \begin{equation}
      C = c - A\left( A^{\mathrm{T}}A \right) ^{-1}A^{\mathrm{T}} c - B\left( B^{\mathrm{T}}B \right) ^{-1}B^{\mathrm{T}} c,
    \end{equation}
    下面验证它垂直于$A$和$B$:
    令$Q = [A\ B]$, 则
    \begin{equation}
      Q^{\mathrm{T}}Q  =\begin{bmatrix}
       A^{\mathrm{T}}\\
       B^{\mathrm{T}}\\
      \end{bmatrix}
      \begin{bmatrix}
       A & B\\
      \end{bmatrix}
      = \begin{bmatrix}
       A^{\mathrm{T}}A & 0\\
       0 & B^{\mathrm{T}}B\\
      \end{bmatrix}
    \end{equation}
    所以
    \begin{equation}
      \left( Q^{\mathrm{T}}Q \right) ^{-1} = \begin{bmatrix}
       \frac{1}{A^{\mathrm{T}}A} & 0\\
       0 & \frac{1}{B^{\mathrm{T}}B}\\
      \end{bmatrix}
    \end{equation}
    得到
    \begin{equation}
      Q\left( Q^{\mathrm{T}}Q \right) ^{-1}Q^{\mathrm{T}} =
      \begin{bmatrix}
       A & B\\
      \end{bmatrix}
      \begin{bmatrix}
       \frac{1}{A^{\mathrm{T}}A} & 0\\
       0 & \frac{1}{B^{\mathrm{T}}B}\\
      \end{bmatrix}
      \begin{bmatrix}
       A^{\mathrm{T}}\\
       B^{\mathrm{T}}\\
      \end{bmatrix}
      =
      A\left( A^{\mathrm{T}}A \right) ^{-1} A^{\mathrm{T}} + B \left( B^{\mathrm{T}}B \right) ^{-1} B^{\mathrm{T}}.
    \end{equation}
    所以上面的$C$等价于将$c$减去$A,B$面内的投影, 自然$C$是垂直于$A,B$的.

    \item 构造$D$垂直于$A,B,C$,
    \begin{equation}
      D = d - A\left( A^{\mathrm{T}}A \right) ^{-1}A^{\mathrm{T}} d - B\left( B^{\mathrm{T}}B \right) ^{-1} B^{\mathrm{T}} d - C\left( C^{\mathrm{T}}C \right)^{-1}C^{\mathrm{T}} d. 
    \end{equation}
    $$
    \vdots
    $$
\end{enumerate}
最终可以得到一组正交的基向量, 之后将它们归一化就得到了正交归一基向量.