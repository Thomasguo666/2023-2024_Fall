% !TeX root = 线性代数.tex

行列式的几个性质:
\begin{itemize}
    \item $\delta \left( A \right) \neq  0$, 当且仅当:
    \begin{enumerate}
        \item $A$是可逆的
        \item $A$满秩
        \item $A$的列向量线性无关
        \item $Ax=0$只有零解
        \item $A$可以行约化为单位矩阵
    \end{enumerate}
    \item $\delta \left( A \right) = 0$, 当且仅当:
    \begin{enumerate}
        \item $A$不可逆
        \item $A$不满秩
        \item $A$的列向量线性相关
        \item $Ax=0$有非零解
        \item $A$不能行约化为单位矩阵
    \end{enumerate}
\end{itemize}

\begin{proposition}
    上文定义的行列式满足 $\det I = 1$.
\end{proposition}
\begin{proof}
    可以用递归定义验证.
\end{proof}

\begin{proposition}
    上文定义的行列式作用在行向量上是线性的. 
    
    \vspace{.5em}
    设$D = \begin{bmatrix}
     \vdots\\
     c a_k + c' b_k\\
     \vdots\\
    \end{bmatrix}$, $A = \begin{bmatrix}
     \vdots\\
     a_k\\
     \vdots\\
    \end{bmatrix}$, $B = \begin{bmatrix}
     \vdots\\
     b_k\\
     \vdots\\
    \end{bmatrix}$, 那么
    \begin{equation}
        \det D = c \det A + c' \det B.
    \end{equation}
\end{proposition}

\begin{proof}
    根据上面的递归定义, 我们把$D$, $A$, $B$ 的行列式展开为
    \begin{equation}
        \det D = \sum \left( - \right) ^{\mu + 1}d_{\mu 1} \det D_{\mu 1},
    \end{equation}
    \begin{equation}
      \det A = \sum \left( - \right) ^{\mu + 1}a_{\mu 1} \det A_{\mu 1},
    \end{equation}
    \begin{equation}
      \det B = \sum \left( - \right) ^{\mu + 1}b_{\mu 1} \det B_{\mu 1}.
    \end{equation}
    我们需要证明
    \begin{equation}
      d_{\mu 1} \det D_{\mu 1} = c a_{\mu 1} \det A_{\mu 1} + c' b_{\mu 1} \det B_{\mu 1}. 
    \end{equation}
    
    分情况讨论, 如果$\mu = k$, 那么三个余子式是一样的, 而前面的系数满足$d_{k 1} = c a_{k 1} + c' b_{k 1}$所以等式成立. 
    
    如果$\mu \neq k$, 那么$d_{\mu 1} = a_{\mu 1} = b_{\mu 1}$, 有递归假设
    \begin{equation}
      \det D_{\mu 1} = c \det A_{\mu 1} + c' \det B_{\mu 1},
    \end{equation}
    那么
    \begin{equation}
        d_{\mu 1} \det D_{\mu 1} = c a_{\mu 1} \det A_{\mu 1} + c' b_{\mu 1} \det B_{\mu 1}.
    \end{equation}
\end{proof}

\begin{proposition}
    如上定义的行列式, 如果$A$有两行是一样的, 那么$\det A = 0$.
\end{proposition}

\begin{proof}
    不妨设$A$的第$k$行和第$k+1$行是一样的. 那么有
    \begin{equation}
      a_{k 1} = a_{k+1 ,1}, \quad \det A_{k 1} = \det A_{k+1 ,1}.
    \end{equation}
    由递归假设, $\det A _{i 1} = 0, \ i \neq k, k+1$.
    \begin{equation}
      \det A = \left( - \right) ^{k + 1} a_{k 1} \det A_{k 1} + \left( - \right) ^{k + 2} a_{k+1 ,1} \det A_{k+1 ,1} = 0.
    \end{equation}
\end{proof}

一些特殊矩阵的行列式:
\begin{itemize}
    \item 对角矩阵的行列式等于对角线上元素的乘积.
    \begin{equation}
      \det \begin{bmatrix}
        d_1 & & & \\
        & d_2 & & \\
        & & \ddots & \\
        & & & d_n
        \end{bmatrix} = d_1 d_2 \cdots d_n.
    \end{equation}

    \item 上三角矩阵的行列式等于对角线上元素的乘积.
    \begin{equation}
        \det \begin{bmatrix}
            d_1 & * & * & * \\
            & d_2 & * & * \\
            & & \ddots & * \\
            & & & d_n
            \end{bmatrix} = d_1 d_2 \cdots d_n.
    \end{equation}

    \item 下三角矩阵的行列式等于对角线上元素的乘积.
    \begin{equation}
        \det \begin{bmatrix}
            d_1 & & & \\
            * & d_2 & & \\
            * & * & \ddots & \\
            * & * & * & d_n
            \end{bmatrix} = d_1 d_2 \cdots d_n.
    \end{equation}
\end{itemize}

\begin{proposition}
    行列式可以用任意一行或者一列展开.

    用行展开, 用$A$的第$i$行展开, 有
    \begin{equation}
      \det A = \sum_{j=1}^{n} \left( - \right) ^{i + j} a_{i j} \det A_{i j}.
    \end{equation}

    用列展开, 用$A$的第$j$列展开, 有
    \begin{equation}
      \det A = \sum_{i=1}^{n} \left( - \right) ^{i + j} a_{i j} \det A_{i j}.
    \end{equation}
\end{proposition}

\begin{example}
    计算
    \begin{equation}
      \det A = \det
      \begin{bmatrix}
       1 & 2 & 3\\
       2 & 3 & 1\\
       1 & 0 & 2\\
      \end{bmatrix}
    \end{equation}
    
    我们对第$1$列展开,
    \begin{equation}
        \det A = 1 \det \begin{bmatrix}
         3 & 1\\
         0 & 2\\
        \end{bmatrix} - 2 \det \begin{bmatrix}
         2 & 3\\
         0 & 2\\
        \end{bmatrix} + 1 \det \begin{bmatrix}
            2 & 3\\
            3 & 1\\
        \end{bmatrix} = 1 \cdot 6 - 2 \cdot 4 + 1 \cdot \left( -7 \right) = -9.
    \end{equation}
    对第$1$行展开
    \begin{equation}
      \det A = 1 \cdot 6 - 2\cdot 3 + 3 \cdot \left( -3 \right) = -9.
    \end{equation}
    对第$2$行展开
    \begin{equation}
        \det A = - 2 \cdot 4 + 3 \cdot \left( -1 \right) - 1 \cdot \left( -2 \right) = -9.
    \end{equation}
\end{example}

行列式的置换定义:

\begin{equation}
  \det A = \sum_{p} \operatorname{sgn} \left( p \right) a_{1,\  p_1} a_{2,\  p_2} \cdots a_{n,\  p_n},
\end{equation}
其中的$p$为一个$n$阶置换$p\colon \{ 1,2, \cdots , n \} \to \{ 1,2, \cdots , n \} $的一一映射, $\operatorname{sgn} \left( p \right)$为置换$p$的符号.

\begin{example}
    三阶置换群的群元:
    
    \begin{center}
        \begin{tabular}{|c|c|c|}
            \hline
            1 & 2 & 3\\
            \hline
            1 & 2 & 3\\
            \hline
        \end{tabular}
        \quad
        \begin{tabular}{|c|c|c|}
            \hline
            1 & 2 & 3\\
            \hline
            2 & 1 & 3\\
            \hline
        \end{tabular}
        \quad
        \begin{tabular}{|c|c|c|}
            \hline
            1 & 2 & 3\\
            \hline
            3 & 2 & 1\\
            \hline
        \end{tabular}
       
        \vspace{1em}
        \begin{tabular}{|c|c|c|}
            \hline
            1 & 2 & 3\\
            \hline
            1 & 3 & 2\\
            \hline
        \end{tabular}
        \quad
        \begin{tabular}{|c|c|c|}
            \hline
            1 & 2 & 3\\
            \hline
            2 & 3 & 1\\
            \hline
        \end{tabular}
        \quad
        \begin{tabular}{|c|c|c|}
            \hline
            1 & 2 & 3\\
            \hline
            3 & 1 & 2\\
            \hline
        \end{tabular}
    \end{center}
\end{example}

\begin{example}
    用置换的方法计算三阶行列式.
    \begin{equation}
        \begin{gathered}
            \det \begin{bmatrix}
             a_{11} & a_{12} & a_{13}\\
             a_{21} & a_{22} & a_{23}\\
             a_{31} & a_{32} & a_{33}\\
            \end{bmatrix} = 
              \sum_{p} \operatorname{sgn} \left( p \right) a_{1,\  p_1} a_{2,\  p_2} a_{3,\  p_3}
              \\
              =
              a_{11} a_{22} a_{33} - a_{11} a_{23} a_{32} - a_{12} a_{21} a_{33} \\ + a_{12} a_{23} a_{31} + a_{13} a_{21} a_{32} - a_{13} a_{22} a_{31}.
          \end{gathered}
    \end{equation}
\end{example}

\subsection{行列式的应用}
\subsubsection{克拉默(Cramer)法则求解线性方程组}
\begin{theorem}
    设$A$是一个$n$阶方阵, $\det \left( A \right) \neq 0$, $b$是一个$n$维列向量, 那么线性方程组
    \begin{equation}
      Ax = b
    \end{equation}
    有唯一解
    \begin{equation}
      x_i = \frac{\det \left( B_i \right)}{\det \left( A \right)}, \quad i = 1,2, \cdots , n,
    \end{equation}
    其中$B_i$是把$A$的第$i$列换成$b$得到的矩阵,
    \begin{equation}
      A = \begin{bmatrix} v_1, v_2, \cdots, v_n \end{bmatrix}, B_i = \begin{bmatrix} v_1, v_2, \cdots, b, \cdots, v_n \end{bmatrix}.
    \end{equation}
\end{theorem}

\begin{proof}
    对于矩阵$A = \begin{bmatrix} v_1, v_2, \cdots, v_n \end{bmatrix}$, 考虑下面的矩阵方程
    \begin{equation}
      A \begin{bmatrix} E_1,  \cdots, x, \cdots,  E_n \end{bmatrix} = \begin{bmatrix} v_1, v_2, \cdots, b, \cdots, v_n \end{bmatrix}.
    \end{equation}
    两边取行列式, 有
    \begin{equation}
      \det \left( A \right) \det \left( \begin{bmatrix} E_1,  \cdots, x, \cdots,  E_n \end{bmatrix} \right) = \det \left( \begin{bmatrix} v_1, v_2, \cdots, b, \cdots, v_n \end{bmatrix} \right).
    \end{equation}
    我们需要计算上式左侧的行列式, 不难发现,
    \begin{equation}
      \det
      \begin{bmatrix}
         1 & 0 & \cdots & x_1 & \cdots & 0\\
            0 & 1 & \cdots & x_2 & \cdots & 0\\
            \vdots & \vdots & \ddots & \vdots & \ddots & \vdots\\
            0 & 0 & \cdots & x_i & \cdots & 0\\
            \vdots & \vdots & \ddots & \vdots & \ddots & \vdots\\
            0 & 0 & \cdots & x_n & \cdots & 1\\
      \end{bmatrix} = x_i.
    \end{equation}
    
    所以
    \begin{equation}
        \left( \det A \right) x_i = \det \left( B_i \right).
    \end{equation}
    即
    \begin{equation}
      x_i = \frac{\det \left( B_i \right)}{\det \left( A \right)}.
    \end{equation}
\end{proof}

\subsubsection{用行列式求逆的公式}
\begin{theorem}
    我们构造一个代数余子式矩阵
    \begin{equation}
        M_{ij} = \left( - \right) ^{i + j} \det A_{ij},
    \end{equation}
    
    那么$A$的逆矩阵为
    \begin{equation}
        A^{-1} = \frac{1}{\det A} M^{\mathrm{T}}.
    \end{equation}
    其中的$M^{\mathrm{T}}$称为$A$的伴随矩阵, 记为$A^{*}$.
\end{theorem}


\begin{example}
    求$A$的逆矩阵, 
    \begin{equation}
      A = \begin{bmatrix}
       1 & 2 & 3\\
       4 & 3 & 1\\
       2 & 1 & 0\\
      \end{bmatrix}.
    \end{equation}

    \begin{equation}
      M = \begin{bmatrix}
       -1 & 2 & -2\\
       3 & -6 & 3\\
       -7 & 11 & -5\\
      \end{bmatrix},
    \end{equation}
    \begin{equation}
      A^{-1} = - \frac{1}{3} \begin{bmatrix}
       -1 & 3 & -7\\
       2 & -6 & 11\\
       -2 & 3 & -5\\
      \end{bmatrix}.
    \end{equation}
\end{example}

\begin{proof}[证明逆矩阵公式]
    求$A$的逆, 假设$A^{-1} = \begin{bmatrix} w_1, w_2, \cdots, w_n \end{bmatrix}$, 那么由于$A A^{-1} = I$,
    \begin{equation}
      A \begin{bmatrix} w_1, w_2, \cdots, w_n \end{bmatrix} = I
      \implies 
      \begin{bmatrix} Aw_1, Aw_2, \cdots, Aw_n \end{bmatrix} = \begin{bmatrix} E_1, E_2, \cdots, E_n \end{bmatrix}.
    \end{equation}
    
    需要求解线性方程组, 
    \begin{equation}
      A w_i = E_i.
    \end{equation}

    由克拉默法则,
    \begin{equation}
      w_{ji} = \frac{\det \left( B_j \right)}{\det \left( A \right)}.
    \end{equation}

    由于$B_i$是把$A$的第$i$列换成$E_i$得到的矩阵, 所以
    \begin{equation}
      \det \left( B_j \right) = \left( - \right) ^{i + j} \det \left( A_{i j} \right) = M_{ij}.
    \end{equation}
\end{proof}