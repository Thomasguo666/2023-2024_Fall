% !TeX root = 线性代数.tex

\subsection{矩阵的乘法的应用}

% 1
对于$n \times n$的方阵$A$, 可以利用矩阵的乘法来定义它的逆矩阵$A^{-1}$
\begin{definition}
    $A^{-1}$称为$A$的逆矩阵, 如果$A^{-1}$满足
    \begin{equation}
      A^{-1} A = I_{n \times n}, \text{且} A A^{-1} = I_{n \times n}.
    \end{equation}
\end{definition}

\begin{proposition}
    \begin{equation}
      I_{n \times n} A = A I_{n \times n} = A
    \end{equation}
\end{proposition}
\begin{proof}
    根据乘法的定义, 不难证明. 
\end{proof}

\subsubsection{逆矩阵的一些性质}

\begin{proposition}
    一个矩阵的逆矩阵不一定存在
\end{proposition}
\begin{example}
    非平凡的例子$\begin{bmatrix}
     0 & 1\\
     0 & 0\\
    \end{bmatrix}$.
    \begin{proof}
        假设存在$A^{-1} = \begin{bmatrix}
            a & b\\
            c & d\\
           \end{bmatrix}$, 那么$A^{-1}$满足
           \begin{equation}
             \begin{bmatrix}
              0 & 1\\
              0 & 0\\
             \end{bmatrix}
             \begin{bmatrix}
              a & b\\
              c & d\\
             \end{bmatrix}
             =
             \begin{bmatrix}
              1 & 0\\
              0 & 1\\
             \end{bmatrix}
           \end{equation}
           左侧
           \begin{equation}
             = \begin{bmatrix}
              c & d\\
              0 & 0\\
             \end{bmatrix} \neq I_{2\times 2}
           \end{equation}
    \end{proof}
\end{example}

\begin{proposition}
    如果逆矩阵存在, 它是唯一的.
\end{proposition}
\begin{proof}
    假设$A$有两个逆矩阵$B,C$,则
    \begin{equation}
      AB = BA = AC = CA = I_{n \times n}
    \end{equation}
    所以
    \begin{equation}
      B = B(AC) = (BA)C = C.
    \end{equation}
\end{proof}

\begin{proposition}
    若一个矩阵存在左逆$L$, 满足$LA = I_{n \times n}$, 那么矩阵$A$的逆矩阵存在, 且等于$L$.\footnote{将在后面证明.}
\end{proposition}

\begin{proposition}
    如果$A$的逆为$A^{-1}$, $B$的逆为$B^{-1}$, 则
    \begin{equation}
      (AB)^{-1} = B^{-1} A^{-1}
    \end{equation}
\end{proposition}
\begin{proof}
    \begin{equation}
      \begin{aligned}
        &(AB)B^{-1}A^{-1}
        \\
        & = A(B B^{-1})A^{-1}
        \\
        & = A A^{-1}
        \\
        & = I_{n \times n} .
      \end{aligned}
    \end{equation}
    证明也可以推广到一般情况.
\end{proof}

\begin{proposition}
    $2\times 2$矩阵$A = \begin{bmatrix}
     a & b\\
     c & d\\
    \end{bmatrix}$的逆矩阵
    \begin{equation}
      A^{-1} = \frac{1}{ad -bc} \begin{bmatrix}
       d & -b\\
       -c & a\\
      \end{bmatrix}
    \end{equation}
\end{proposition}

\begin{proposition}
    对角矩阵$D = \begin{bmatrix}
        d_1&0&\cdots&0\\
        0&d_2&\cdots&0\\
        \vdots&\vdots&\ddots&\vdots\\
        0&0&\cdots&d_n
    \end{bmatrix}$的逆为
    \begin{equation}
      D^{-1}
      =
      \begin{bmatrix}
        \frac{1}{d_1}&0&\cdots&0\\
        0&\frac{1}{d_2}&\cdots&0\\
        \vdots&\vdots&\ddots&\vdots\\
        0&0&\cdots&\frac{1}{d_n}
      \end{bmatrix}
    \end{equation}
    这意味着$D^{-1}$存在当且仅当对角元素都不为零!
\end{proposition}

\subsubsection{线性组合的矩阵乘法表示}
线性组合
\begin{equation}
  x_1\vec{v}_1 + x_2 \vec{v}_2 + \cdots +x_n \vec{v}_n,
\end{equation}
引入两个矩阵
\begin{equation}
  A = \begin{bmatrix} \vec{v}_1, \vec{v}_2, \cdots, \vec{v}_n \end{bmatrix}, \quad 
  X = \begin{bmatrix} 
  x_1 \\ 
  x_2 \\ 
  \vdots \\ 
  x_n 
  \end{bmatrix}
\end{equation}
其中$A$为$m \times n$矩阵, $X$为$n \times 1$矩阵. 线性组合的矩阵表示为
$AX$.

\subsubsection{矩阵方程}
方程为$AX = b$. 这个方程的解的性质取决于$A$中的向量$\begin{bmatrix} \vec{v}_1, \vec{v}_2, \cdots, \vec{v}_n \end{bmatrix}$和向量$b$的关系.

我们把这个方程写成分量的形式
\begin{equation}
  \begin{bmatrix}
    a_{11}&a_{12}&\cdots&a_{1n}\\
    a_{21}&a_{22}&\cdots&a_{2n}\\
    \vdots&\vdots&\ddots&\vdots\\
    a_{m 1}&a_{m 2}&\cdots&a_{m n}
  \end{bmatrix}
  \begin{bmatrix} 
  x_1 \\ 
  x_2 \\ 
  \vdots \\ 
  x_n 
  \end{bmatrix}
  =
  \begin{bmatrix} 
  b_1 \\ 
  b_2 \\ 
  \vdots \\ 
  b_m
  \end{bmatrix}.
\end{equation}

采用高斯(Gauss)消元法.


把第$2$个方程到第$m$个方程中的$x_1$消掉. 把第$i$个方程变为
\begin{equation}
  \text{方程}(i) - \frac{a_{i 1}}{a_{11}} \times \text{方程}(1)
\end{equation}
于是方程的增广矩阵变为
\begin{equation}
  \left[
    \begin{matrix}
        a_{11}&\cdots&a_{1n}'\\
        0&\cdots&a_{2n}'\\
        \vdots&\ddots&\vdots\\
        0&\cdots&a_{m n}'
    \end{matrix}
    \ 
  \middle|
    \ 
    \begin{matrix} 
    b_1 \\ 
    b_2 \\ 
    \vdots \\ 
    b_m
    \end{matrix}
  \right] 
\end{equation}