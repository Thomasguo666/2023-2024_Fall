% !TeX root = 线性代数.tex
\subsubsection{$QR$分解}
正交归一化的过程, 给出了一个可逆矩阵的$QR$分解,
\begin{equation}
  A = QR
\end{equation}
其中 $Q$是正交矩阵, $Q^{-1} = Q^{\mathrm{T}}$, $R$是上三角矩阵.

正交归一化当中,
\begin{gather}
    A = a
    \\
    B = b - \frac{A^{\mathrm{T}}b}{A^{\mathrm{T}}A}A
    \\
    C = c - \frac{A^{\mathrm{T}}c}{A^{\mathrm{T}}A}A - \frac{B^{\mathrm{T}}c}{B^{\mathrm{T}}B}B
\end{gather}
整个过程可以用矩阵乘法表示. 正交归一基为
\begin{equation}
  Q = \left[ \frac{A}{|A|} , \frac{B}{|B|}, \cdots   \right] 
\end{equation}
有
\begin{equation}
  Q = AR
\end{equation}
$R$是一个上三角, 可以把原矩阵写为$A = Q R^{-1}$.

\section{行列式}

\subsection{行列式的定义和唯一性}
我们之前讨论了秩$r\colon M_{n \times n} \to \mathbb{Z}_{+}$, 它在初等行变换下不变.

我们引入行列式
\begin{equation}
  \delta\colon M_{n \times n} \to \mathbb{R}
\end{equation}
是一个实数.

这个函数满足三个性质:
\begin{itemize}
    \item 作用在单位阵上, $\delta(I) = 1$
    \item 作用在行向量上是线性的
    \begin{equation}
      \delta \begin{bmatrix}
       \vdots\\
       c A_i + c' B_i\\
       \vdots\\
      \end{bmatrix}
      =
      c \delta 
      \begin{bmatrix}
       \vdots\\
       A_i\\
       \vdots\\
      \end{bmatrix}
      +
      c'\delta
      \begin{bmatrix}
       \vdots\\
       B_i\\
       \vdots\\
      \end{bmatrix}
    \end{equation}

    \item 如果$A$有两行是一样的, 那么行列式为零.
\end{itemize}

下面研究行列式在初等变换下的性质
\begin{itemize}
    \item 倍加变换: $\delta (A') = \delta (A)$.
    \begin{proof}
        用到了性质三
        \begin{equation}
          \delta (A') = \delta \begin{bmatrix}
           \vdots\\
           w_j\\
           \vdots\\
           w_i + a w_j\\
           \vdots\\
          \end{bmatrix}
          = 
          \delta \begin{bmatrix}
            \vdots\\
            w_j\\
            \vdots\\
            w_i \\
            \vdots\\
           \end{bmatrix}
          +
          a \delta 
          \begin{bmatrix}
           \vdots\\
           w_j\\
           \vdots\\
           w_j\\
           \vdots\\
          \end{bmatrix}
          =\delta (A)
        \end{equation}
    \end{proof}
    \item 换行变换: $\delta(A') = -\delta (A)$.
    \begin{proof}
        \begin{equation}
            0=
            \delta \begin{bmatrix}
                \vdots\\
                w_j + w_i\\
                \vdots\\
                w_i + w_j\\
                \vdots\\
               \end{bmatrix}
            = \delta
            \begin{bmatrix}
                \vdots\\
                w_j\\
                \vdots\\
                w_i\\
                \vdots\\
               \end{bmatrix}  
            +\delta \begin{bmatrix}
                \vdots\\
                w_i\\
                \vdots\\
                w_j\\
                \vdots\\
               \end{bmatrix}
            = \delta (A') + \delta (A).
        \end{equation}
    \end{proof}
    \item 倍乘变换$\delta (A') = c \delta (A)$. 由线性性可得.
\end{itemize}

初等矩阵的行列式
\begin{itemize}
    \item $\delta (E_{ij}(a)) = 1$
    \begin{proof}
        令$I = \begin{bmatrix} 
        w_1 \\ 
        w_2 \\ 
        \vdots \\ 
        w_n 
        \end{bmatrix}$, 则初等矩阵可以写为
        \begin{equation}
          E_{ij}(a) =  \begin{bmatrix}
           \vdots\\
           w_j\\
           \vdots\\
           w_i + a w_j\\
           \vdots\\
          \end{bmatrix}.
        \end{equation}
        于是
        \begin{equation}
            \delta(E) = 
            \delta \begin{bmatrix}
                \vdots\\
                w_j\\
                \vdots\\
                w_i + a w_j\\
                \vdots\\
               \end{bmatrix}
               = 
               \delta \begin{bmatrix}
                 \vdots\\
                 w_j\\
                 \vdots\\
                 w_i \\
                 \vdots\\
                \end{bmatrix}
               +
               a \delta 
               \begin{bmatrix}
                \vdots\\
                w_j\\
                \vdots\\
                w_j\\
                \vdots\\
               \end{bmatrix}
               =\delta (I) = 1.
        \end{equation}
    \end{proof}

    \item 换行$\delta (P_{ij})  =-1$.证明同上, 把这个矩阵写成单位阵的换行即可.
    \item 倍乘$\delta (S_{i}(c)) = c$. 证明利用线性性.
\end{itemize}

总结上面的结论, 我们可以发现, 对于初等矩阵$E$, 
\begin{equation}
  \delta (EA) = \delta (A).
\end{equation}
并且, 如果一个矩阵某一行为零, 那么行列式为零.

\begin{proposition}
    满足行列式定义的三个性质的函数是唯一的.
\end{proposition}

\begin{proof}
    对于任意矩阵$A$, 通过初等变换可以变为一个行约化阶梯形式.
    这分为两种情况.

    如果$A'$为单位矩阵, 那么
    \begin{equation}
      \delta (A)  = \frac{1}{\delta(E_p) \cdots \delta(E_1)}.
    \end{equation}

    如果$A'$不是单位矩阵, 那么$A'$的最后一行为零, 则
    \begin{equation}
      \delta (A) = 0.
    \end{equation}
\end{proof}

行列式满足的一个重要性质: 
\begin{equation}
  \delta (AB) = \delta (A) \delta (B).
\end{equation}


\begin{proof}
  找到变换使得$A$变为行约化阶梯形式$A'$
  \begin{equation}
    A' = \left( E_p \cdots E_1 \right) A,
  \end{equation}
  如果$A'$为单位矩阵, 那么$\delta (A') = 1$
  \begin{equation}
    \delta (B) = \delta (A'B) = \delta (E_p \cdots E_1 A B) = \delta (E_p) \cdots \delta (E_1) \delta (AB),
  \end{equation}
  所以
  \begin{equation}
    \delta (B) = \frac{\delta(AB)}{\delta (A)} \implies \delta (AB) = \delta (A) \delta (B).
  \end{equation}

  如果$A'$不是单位矩阵, 那么$\delta (A') = 0$, $AB$也不是满秩的, 所以
  \begin{equation}
    \delta (AB) = 0 = \delta (A) \delta (B).
  \end{equation}
\end{proof}

\subsection{行列式的递归定义}
\begin{definition}
  余矩阵: $A_{ij}$: 把$A$的第$i$行第$j$列去掉, 得到一个$ \left( n-1 \right) \times  \left( n-1 \right) $的矩阵.
\end{definition}
\begin{example}
  一个矩阵$A = \begin{bmatrix}
   2 & 3 & 4\\
   5 & 6 & 7\\
   8 & 9 & 10\\
  \end{bmatrix}$,
  \begin{equation}
    A_{11} = \begin{bmatrix}
     6 & 7\\
     9 & 10\\
    \end{bmatrix}
    \quad
    A_{21} = \begin{bmatrix}
     3 & 4\\
     9 & 10\\
    \end{bmatrix}
    \quad
    A_{31} = \begin{bmatrix}
     3 & 4\\
     6 & 7\\
    \end{bmatrix}.
  \end{equation}
\end{example}

\textbf{行列式的递归定义:}
\begin{equation}\boxed{
\det A = a_{11} \det A_{11} - a_{21} \det A_{21} + a_{31} \det A_{31} - \cdots + \left( - \right) ^{1+j} a_{1j} \det A_{1j}. }
\end{equation}

\begin{example}
  $1 \times 1$矩阵行列式 $\det [a] = a$.
  $2 \times 2$矩阵行列式 $\det \begin{bmatrix}
   a & b\\
   c & d\\
  \end{bmatrix} = ad - bc.$
\end{example}