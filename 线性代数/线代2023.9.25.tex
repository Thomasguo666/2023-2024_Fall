% !TeX root = 线性代数.tex
\section{矩阵的初等变换}
下面我们要将初等变换用矩阵乘法来实现.

\begin{itemize}
    \item 倍乘变换: 矩阵$A$的第$i$行乘上$c$, 其他行不变, $S_i(c) A = A'$. $S_i(c)$为将单位矩阵的第$i$个元素换为$c$.
    \begin{equation}
      S_i (c) = \begin{bmatrix}
        \ddots&0&\cdots&0\\
        0&c&\cdots&0\\
        \vdots&\vdots&\ddots&\vdots\\
        0&0&\cdots&1
      \end{bmatrix}
    \end{equation}
    $S^{-1}_i (c)$是可逆的
    \begin{equation}
      S^{-1}_i (c) = \begin{bmatrix}
        \ddots&0&\cdots&0\\
        0&\frac{1}{c}&\cdots&0\\
        \vdots&\vdots&\ddots&\vdots\\
        0&0&\cdots&1
      \end{bmatrix}
    \end{equation}

    \item 消元变换: 
    把第$i$行换成$\text{第$i$行} + a \times \text{第$j$行}$. 它的形式为
    \begin{equation}
      E_{ij}(a) = \begin{bmatrix}
       1 &  &  &  & \\
        & \ddots &  &  & \\
        &  & \ddots &  & \\
        & a &  & \ddots & \\
        &  &  &  & 1\\
      \end{bmatrix}.
    \end{equation}
    其中的$a$位于$E_{ij}(a)$的第$i$行第$j$列, 对角线元素都为$1$.
    
    $E_{ij}(a)$是可逆的
    \begin{equation}
      E_{ij}^{-1}(a) = E_{ij}(-a)
    \end{equation}

    \item 换行变换:
    把第$i$行和第$j$行交换. 用一个矩阵$P_{ij} = (\text{交换单位矩阵的$i,j$列})$来表示.
    它是可逆的,
    \begin{equation}
      P^{-1}_{ij}= P_{ij}.
    \end{equation}
\end{itemize}


\subsection{初等变换的应用}
\paragraph{$LU$分解}
把方矩阵分解成下列形式 
\begin{equation}
  A = LU,
\end{equation}
$L$为下三角矩阵, $U$为上三角矩阵.

方法: 通过初等变换, 把方矩阵变成一个上三角矩阵(如果这个过程不涉及到换行的话)那么有
\begin{equation}
  E_1 E_2 \cdots E_{n} A = U \iff A = E_n ^{-1} \cdots E_2^{-1} E_1^{-1} U
\end{equation}
由于$E_1 ,\cdots ,E_n$都是下三角矩阵, 它们的乘积也是下三角矩阵.

\begin{example}
    对矩阵$A$做$LU$分解.
    \begin{equation}
      A = \begin{bmatrix}
       3 & -7 & -2 & 2\\
       -3 & 5 & 1 & 0\\
       6 & -4 & 0 & -5\\
       -9 & 5 & -5 & 12\\
      \end{bmatrix}
    \end{equation}
    做操作
    \begin{equation}
        \begin{gathered}
            A \xlongrightarrow{E_{41}(3)E_{31}(-2)E_{21}(1)} 
            \begin{bmatrix}
             3 & -7 & -2 & 2\\
             0 & -2 & -1 & 2\\
             0 & 10 & 4 & -9\\
             0 & -16 & -11 & 18\\
            \end{bmatrix}
            \\
            \xlongrightarrow{E_{42}(-8)E_{32}(5)}
            \begin{bmatrix}
             3 & -7 & -2 & 2\\
             0 & -2 & -1 & 2\\
             0 & 0 & -1 & 1\\
             0 & 0 & -3 & 2\\
            \end{bmatrix}
            \\
            \xlongrightarrow{E_{43}(-3)}
            \begin{bmatrix}
             3 & -7 & -2 & 2\\
             0 & -2 & -1 & 2\\
             0 & 0 & -1 & 1\\
             0 & 0 & 0 & 1\\
            \end{bmatrix}
        \end{gathered}
    \end{equation}
    用矩阵乘法写起来
    \begin{equation}
      E_{43}(-3)E_{42}(-8)E_{32}(5)E_{41}(3)E_{31}(-2)E_{21}(1)A = U
    \end{equation}
    即
    \begin{equation}
      L = E_{21}(-1) E_{31}(2)E_{41}(-3)E_{32}(-5)E_{42}(8)E_{43}(-3)
    \end{equation}
    计算可得
    \begin{equation}
      L = \begin{bmatrix}
       1 & 0 & 0 & 0\\
       -1 & 1 & 0 & 0\\
       2 & -5 & 1 & 0\\
       -3 & 8 & 3 & 1\\
      \end{bmatrix}.
    \end{equation}
\end{example}

如果$U$的对角线都不为零, 那么
\begin{equation}
  U = D U' , \quad (U'\text{对角线都为1})
\end{equation}

\paragraph{用初等变换求逆(Gauss-Jordan)}
原理: 先找到一组初等变换来使得矩阵$A$变为单位矩阵,
\begin{equation}
  (E_p \cdots E_1)A = I.
\end{equation}
$A$的逆可以这样求解:
\begin{equation}
  AB = I
\end{equation}
在两边同时做行变换.
\begin{equation}
  (E_p\cdots E_1) AB = (E_p\cdots E_1)I
\end{equation}
即
\begin{equation}
  B = I B = (E_p \cdots E_1)I
\end{equation}
在实际操作时, 考虑增广矩阵$[A|I]$, 做初等变换, 变为$[I|A^{-1}]$
\begin{example}
    \begin{equation}
      A = \begin{bmatrix}
       2 & -1 & 0\\
       -1 & 2 & -1\\
       0 & -1 & 2\\
      \end{bmatrix}
    \end{equation}
    做初等变换
    \begin{equation}
      \begin{gathered}
        \relax [ A | I ] = \left[ \begin{matrix}
            2 & -1 & 0\\
            -1 & 2 & -1\\
            0 & -1 & 2\\
           \end{matrix}
           \ 
           \middle|
           \ 
           \begin{matrix}
            1 & 0 & 0\\
            0 & 1 & 0\\
            0 & 0 & 1\\
           \end{matrix}
           \right] 
           \\
           \xlongrightarrow{E_{12}(\frac{2}{3})E_{23}(\frac{3}{4})E_{32}(\frac{2}{3})E_{21}(\frac{1}{2})}
           \left[ \begin{matrix}
            2 & 0 & 0\\
            0 & \frac{3}{2} & 0\\
            0 & 0 & \frac{4}{3}\\
           \end{matrix}
           \ 
           \middle|
           \ 
           \begin{matrix}
            \frac{3}{2} & 1 & \frac{1}{2}\\
            \frac{3}{4} & \frac{3}{2} & \frac{3}{4}\\
            \frac{1}{3} & \frac{2}{3} & 1\\
           \end{matrix}
           \right] 
           \\
           \xlongrightarrow{S_3(\frac{3}{4})S_2(\frac{2}{3})S_1(\frac{1}{2})}
           \left[ \begin{matrix}
            1 & 0 & 0\\
            0 & 1 & 0\\
            0 & 0 & 1\\
           \end{matrix}
           \ 
           \middle|
           \ 
           \begin{matrix}
            \frac{3}{4} & \frac{1}{2} & \frac{1}{4}\\
            \frac{1}{2} & 1 & \frac{1}{2}\\
            \frac{1}{4} & \frac{1}{2} & \frac{3}{4}\\
           \end{matrix}
           \right]
      \end{gathered}
    \end{equation}
    最终的增广矩阵右侧就是$A$的逆,
    \begin{equation}
      A^{-1} = \begin{bmatrix}
       \frac{3}{4} & \frac{1}{2} & \frac{1}{4}\\
       \frac{1}{2} & 1 & \frac{1}{2}\\
       \frac{1}{4} & \frac{1}{2} & \frac{3}{4}\\
      \end{bmatrix}
    \end{equation}
\end{example}

\paragraph{行约化阶梯形及一般线性方程组的一般解}
\subparagraph{行约化阶梯形式}
\begin{itemize}
    \item 如果第$i$行都是零, 那么对于$j>i$行都是零.
    
    \item 如果第$i$行不都是零, 那么第一个非零元素为$1$, 称为主元.
    
    \item 如果第$(i+1)$行不都是零, 那么这一行的主元在第$i$行的主元右边.

    \item 主元上方的元素都为零.
\end{itemize}


\subparagraph{找到一个矩阵行约化阶梯形的方法}

\begin{itemize}
    \item 找到第一个不为零的列, 然后, 如果需要的话, 做换行操作, 使得这一列的第一个元素不为零. 之后消元操作, 把下方的元素都变为零.
    % \begin{equation}
    %   A \to \begin{bmatrix}
    %    0 & 0 & 1 & * & * & *\\
    %    0 & 0 & 0 &  &  & \\
    %    0 & 0 & 0 &  & \text{\Large$B$} & \\
    %    0 & 0 & 0 &  &  & \\
    %   \end{bmatrix}
    % \end{equation}
    % 换了一种方式
    \begin{equation}
      A \to  
      \left[ \begin{matrix}
       0 & 0 & 1\\
       0 & 0 & 0\\
       0 & 0 & 0\\
       0 & 0 & 0\\
      \end{matrix}
      \quad
      \begin{matrix}
       \begin{matrix}
        *\  & *\  & *\ \\
       \end{matrix}\\
       \boxed{\begin{matrix}
          &  & \\
          & \text{\ \Large $B$ } & \\
          &  & \\
        \end{matrix}}
      \end{matrix}\ 
      \right] 
    \end{equation}

    \item 对子矩阵$B$做同样的操作.
    
    \item 把主元上方的元素变为零.
\end{itemize}


\begin{example}\label{reduction}
    行约化阶梯形式:
    \begin{equation}
      \begin{gathered}
        \begin{bmatrix}
         1 & 1 & 2 & 4\\
         1 & 2 & 2 & 5\\
         1 & 3 & 2 & 6\\
        \end{bmatrix}
        \longrightarrow
        \begin{bmatrix}
         1 & 1 & 2 & 4\\
         0 & 1 & 0 & 1\\
         0 & 2 & 0 & 2\\
        \end{bmatrix}
        \longrightarrow
        \begin{bmatrix}
         1 & 1 & 2 & 4\\
         0 & 1 & 0 & 1\\
         0 & 0 & 0 & 0\\
        \end{bmatrix}
        \longrightarrow
        \begin{bmatrix}
         1 & 0 & 1 & 3\\
         0 & 1 & 0 & 1\\
         0 & 0 & 0 & 0\\
        \end{bmatrix}
      \end{gathered}
    \end{equation}
\end{example}

\subparagraph{行约化阶梯形式的一些名词}
\begin{itemize}
    \item 主元列: 如果该列有一个主元, 该列只有一个非零元素.
    
    \item 自由列: 没有主元之列.
    
    \item 主元变量, 自由变量:
    
    对于一个方程
    \begin{equation}
      Rx = x_1 Y_1 + x_2  Y_2 + \cdots  + x_n Y_n,
    \end{equation}
    如果$Y_i$为自由列, 则$x_i$为自由变量. 如果$Y_i$为主元列, 则$x_i$为主元变量.

    对于例\ref{reduction}, 约化后的方程为
    \begin{equation}
      \begin{bmatrix}
       1 & 0 & 2 & 3\\
       0 & 1 & 0 & 1\\
       0 & 0 & 0 & 0\\
      \end{bmatrix}
      \begin{bmatrix}
       x_1\\
       x_2\\
       x_3\\
       x_4\\
      \end{bmatrix},
    \end{equation}
    其中的$x_1,x_2$为主元变量, $x_3, x_4$为自由变量.
\end{itemize}
