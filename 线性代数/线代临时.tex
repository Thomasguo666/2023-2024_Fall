% !TeX root = 线性代数.tex


\begin{proof}
    找到变换使得$A$变为行约化阶梯形式$A'$
    \begin{equation}
      A' = \left( E_p \cdots E_1 \right) A,
    \end{equation}
    如果$A'$为单位矩阵, 那么$\delta (A') = 1$
    \begin{equation}
      \delta (B) = \delta (A'B) = \delta (E_p \cdots E_1 A B) = \delta (E_p) \cdots \delta (E_1) \delta (AB),
    \end{equation}
    所以
    \begin{equation}
      \delta (B) = \frac{\delta(AB)}{\delta (A)} \implies \delta (AB) = \delta (A) \delta (B).
    \end{equation}

    如果$A'$不是单位矩阵, 那么$\delta (A') = 0$, $AB$也不是满秩的, 所以
    \begin{equation}
      \delta (AB) = 0 = \delta (A) \delta (B).
    \end{equation}
\end{proof}

\subsubsection{行列式的递归定义}
\begin{definition}
    余矩阵: $A_{ij}$: 把$A$的第$i$行第$j$列去掉, 得到一个$ \left( n-1 \right) \times  \left( n-1 \right) $的矩阵.
\end{definition}
\begin{example}
    一个矩阵$A = \begin{bmatrix}
     2 & 3 & 4\\
     5 & 6 & 7\\
     8 & 9 & 10\\
    \end{bmatrix}$,
    \begin{equation}
      A_{11} = \begin{bmatrix}
       6 & 7\\
       9 & 10\\
      \end{bmatrix}
      \quad
      A_{21} = \begin{bmatrix}
       3 & 4\\
       9 & 10\\
      \end{bmatrix}
      \quad
      A_{31} = \begin{bmatrix}
       3 & 4\\
       6 & 7\\
      \end{bmatrix}.
    \end{equation}
\end{example}

行列式的递归定义:
\begin{equation}
  \det A = a_{11} \det A_{11} - a_{21} \det A_{21} + a_{31} \det A_{31} - \cdots + \left( - \right) ^{1+j} a_{1j} \det A_{1j}. 
\end{equation}

\begin{example}
    $1 \times 1$矩阵行列式 $\det [a] = a$.

    $2 \times 2$矩阵行列式 $\det \begin{bmatrix}
     a & b\\
     c & d\\
    \end{bmatrix} = ad - bc.$
\end{example}