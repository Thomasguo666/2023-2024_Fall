% !TeX root = 线性代数.tex

\paragraph{矩阵的转置}
\begin{definition}
    给定一个矩阵$A$,
    \begin{equation}
      \left( A^{\mathrm{T}} \right) _{ij}=A_{ji}.
    \end{equation}
    $A^{\mathrm{T}}$把$A$的行变成列.
\end{definition}

\paragraph{转置的一些性质}
\begin{itemize}
    \item \begin{equation}
        \left( A^{\mathrm{T}} \right) ^{\mathrm{T}} = A.
      \end{equation}
    
    \item \begin{equation}
        \left( AB \right) ^{\mathrm{T}} = B^{\mathrm{T}}A^{\mathrm{T}}
      \end{equation}
    \begin{proof}
          设$A\colon m \times n$, $B\colon k \times n$, 那么
          \begin{equation}
            \left( AB \right) _{ij} = \sum_{l=1}^{k} a_{ik}b_{lj}.
          \end{equation}
          根据转置的定义, 有
          \begin{equation}
            \left( AB \right) ^{\mathrm{T}}_{ij} = \left( AB \right) _{ji} = \sum_{l=1}^{k} a_{jl} b_{li}.
          \end{equation}
      
          另一方面,
          \begin{equation}
            \left( B^{\mathrm{T}}A^{\mathrm{T}} \right) _{ij} = \sum_{l=1}^{k} \left( B^{\mathrm{T}} \right) _{li} \left( A^{\mathrm{T}} \right) _{jl} = \sum_{l=1}^{k} a_{jl} b_{li}.
          \end{equation}
    \end{proof}

    \item \begin{equation}
        \left( A^{\mathrm{T}} \right) ^{-1} = \left( A^{-1} \right) ^{\mathrm{T}}.
      \end{equation}
      \begin{proof}
          因为
          \begin{equation}
            A A^{-1} = I,
          \end{equation}
          两边取转置得到
          \begin{equation}
            \left( A^{-1} \right) ^{\mathrm{T}} A^{\mathrm{T}} = I \implies \left( A^{\mathrm{T}} \right) ^{-1} = \left( A^{-1} \right) ^{\mathrm{T}}.
          \end{equation}
      \end{proof}
\end{itemize}


\paragraph{特殊矩阵}
\begin{itemize}
    \item 对称矩阵: $A^{\mathrm{T}}=A$.
    \begin{equation}
      A_{ij}=A_{ji}.
    \end{equation}
    \begin{example}
        \begin{equation}
          \begin{bmatrix}
           2 & 4\\
           4 & 3\\
          \end{bmatrix}.
        \end{equation}
    \end{example}
    \begin{example}
        \begin{equation}
          \begin{bmatrix}
           2 & 3 & 1\\
           3 & 4 & 3\\
           1 & 3 & 4\\
          \end{bmatrix}.
        \end{equation}
    \end{example}

    \item 反对称矩阵: $A^{\mathrm{T}} = -A$.
    可知, 其对角线都为零.
    
    \begin{example}
        \begin{equation}
          \begin{bmatrix}
           0 & 1 & 1\\
           -1 & 0 & -1\\
           -1 & 1 & 0\\
          \end{bmatrix}.
        \end{equation}
    \end{example}
\end{itemize}


下面我们回到方程$Ax=b$, $A$可以定义四个线性子空间.
\begin{enumerate}
    \item $A$的列向量张成的线性子空间$C(A)$, 它的维数称为$A$的列秩.
    \begin{example}
        对于 $$
        \begin{bmatrix}
         1 & 0 & 0\\
         0 & 0 & 1\\
         0 & 0 & 0\\
        \end{bmatrix}.
        $$
        $C(A) = \begin{bmatrix}
         x_1\\
         x_3\\
         0\\
        \end{bmatrix}$, 是三维空间的$x-y$平面.
    \end{example}

    \item $A$的行向量张成的线性子空间$C(A^{\mathrm{T}})$, 它的维数称为$A$的行秩.
    
    \item $A$的零空间$N(A)$. 线性方程组$Ax=0$的所有解. $N(A)$的维数为$n-r$, $r$为主元数.
    
    \item $A^{\mathrm{T}}$的零空间$N(A^{\mathrm{T}})$. 线性方程组$A^{\mathrm{T}}x = 0$的所有解.
\end{enumerate}



\subsection{线性代数基本定理}
\begin{theorem}
    \begin{equation}
      r_1=r_2=r=r'.
    \end{equation}
\end{theorem}
\begin{proposition}
    初等行变换不改变行秩和列秩.
\end{proposition}
\begin{proof}
 
    \textbf{初等行变换对于行线性空间的影响}

    设矩阵$A = \begin{bmatrix} 
    \vec{w}_1 \\ 
    \vec{w}_2 \\ 
    \vdots \\ 
    \vec{w}_m 
    \end{bmatrix}$.
    把行线性子空间记为$V(w) \subset \mathbb{R}^{n}$.

    做倍加变换之后, 新的向量组为
    \begin{equation}
      w' = (w_1,w_2,\cdots,w_i+a w_j ,\cdots,w_m).
    \end{equation}
    $V(w')$为另一个线性子空间, 但是$V(w)=V(w')$, 因为对于任意的一个向量$w' \in V(w')$, 有
    \begin{equation}
      w'=x_1w_1'+ \cdots +x_m w_m' = x_1 w_1 + \cdots +(x_j+ ax_i)w_j+ \cdots +x_m w_m.
    \end{equation}
    所以有$V(w') \subset V(w)$. 反之, 也有$V(w) \subset V(w')$.

    可得$V(w')=V(w)$.
    \\ \\
    \textbf{初等行变换对于列向量子空间的影响}

    注意到, 初等行变换不改变齐次线性方程组的解, 也就是说
    \begin{equation}
      Ax=0 \iff Bx=0.
    \end{equation}
    其中$B$为$A$初等行变换后的矩阵. 这也就是说, 
    \begin{equation}
      x_1v_1+ \cdots x_n v_n=0 \iff x_1 v_1' + \cdots +x_n v_n ' = 0.
    \end{equation}
    假如$x_1\neq 0$, 那么$v_1$可以用其他向量线性组合表示.

    所以, \textbf{$v$中线性独立的列向量数之和等于$v'$中线性独立的列向量数之和}.

\end{proof}

我们只需考虑行约化阶梯形式$R$, 通过观察$R$的形式, 可以发现
\begin{itemize}
    \item $R$的列秩等于行秩.
    
    \item $R$的行向量子空间及列向量子空间的维数等于主元数目.
\end{itemize}
因为主元列是线性无关的, 自由列都可以用主元列的线性组合表示, 主元行是线性无关的, 而自由行是零. 

\begin{definition}
  矩阵的秩(rank)为列向量子空间$C(A)$的维数. 秩在初等行变换下不变.
\end{definition}