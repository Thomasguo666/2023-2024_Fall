% !TeX root = 费曼(3).tex

and we find
\begin{equation}
  D\left[ R_y \left( \frac{\pi}{2} \right)  \right] = \begin{pmatrix}
   \frac{1}{\sqrt{2}} & \frac{1}{\sqrt{2}}\\
   -\frac{1}{\sqrt{2}} & \frac{1}{\sqrt{2}}\\
  \end{pmatrix}
\end{equation}

or, equally,
\begin{equation}
  D \left[ R_y \left( \pm \frac{\pi}{2} \right)  \right] = \begin{pmatrix}
    \frac{1}{\sqrt{2}} & \pm \frac{1}{\sqrt{2}}\\
    \mp \frac{1}{\sqrt{2}} & \frac{1}{\sqrt{2}}\\
   \end{pmatrix}
\end{equation}

\chapter[概率幅的时间依赖]{Time Dependence of Amplitude}
\section[自由粒子,定态]{Free Particle \& Stationary State}
We have the relativistic mass-shell relation (dispersion relation):
\begin{equation}
  E = \sqrt{\left| \vec{p} \right| ^{2} c^{2} + m^{2} c^{4}}.
\end{equation}

The amplitude of free particle, with given energy $E$ and momentum $\vec{p}$ to be found at position $\vec{r}$ at time $t$ is
\begin{equation}
    \psi_{\vec{p}} \left( t, \vec{x} \right)  = C_{p} \mathrm{e}^{- \mathrm{i} \left( Et - \vec{p}\cdot \vec{x} \right) / \hbar }.
\end{equation}

Note that the phase is invariant under Lorentz transformation
\begin{equation}
  Et - \vec{p}\cdot \vec{x} = \sum_{i=0}^{3} \eta_{\mu\nu}p^{\mu} x^{\nu}
\end{equation}
where the $\eta\mu\nu$ is the Minkowski metric tensor, which is
\begin{equation}
    \eta_{\mu\nu} = \begin{pmatrix}
        -1 & & & \\
         & 1 & & \\
         & & 1 & \\
         & & & 1 \\
    \end{pmatrix}
\end{equation}

In a system free of gravity, observables are invariant under the change of energy zero point.

\section{Breit-Wigner Approximation}
The Breit-Wigner approximation is a method to describe the time dependence of a decaying state. The amplitude of a decaying state is (approximately) 
\begin{equation}
  \psi = C \mathrm{e}^{- \mathrm{i} E_0 t / \hbar} \mathrm{e}^{- \Gamma t / 2 \hbar}.
\end{equation}
And the Fourrier transform of the amplitude is
\begin{equation}
  \tilde{\psi} \left( E \right) = \int_{0}^{\infty} \mathrm{d}t \,  \mathrm{e}^{+ \mathrm{i}  Et} \psi\left( t \right) = \frac{\mathrm{i} C}{E - E_0 + \mathrm{i} \Gamma / 2}.
\end{equation}
then, the power distribution is
\begin{equation}
  \rho \left( E \right) \propto \frac{\Gamma}{\left( E - E_0 \right) ^{2} + \left( \frac{\Gamma}{2} \right) ^{2}}
\end{equation}