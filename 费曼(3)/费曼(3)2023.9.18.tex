% !TeX root = 费曼(3).tex

\chapter{Quantum Mechanic Behaviors}

\section{Two-Slit Interference of Electrons}
Both open: $N \neq N_1(x) + N_2(x)$. In reality, it turns out that the pattern is something like 
\begin{equation}
  N(x) = N_1(x) + N_2(x) + g(x)\sin \left[ \omega(x) x\right],
\end{equation}
where the last term is a interference term,
which satisfies a slow changing condition
\begin{equation}
  \frac{1}{\omega} \frac{\mathrm{d}\omega}{\mathrm{d} x} \ll \omega, \quad \frac{1}{g}\frac{\mathrm{d}g}{\mathrm{d} x} \ll \omega.
\end{equation}

If we lower the power of electron source so that it emits each electron one by one, thus interactions between electrons will not functional, however, the result o  f the experiment recovers. This shows us that the statistical pattern isn't resulted by many-body interactions. So, we come to a ridiculous conclusion, electron must interact with itself passing both hole simultaneously.

Next we build a which-way detector, using a light source for observing whether an electron has passed a hole. The pattern disappears! When we lower the power or enlarging the wavelength, the pattern re-appear gradually.

We have to admit that it is impossible to design an apparatus to determine which hole the electron passes through, 
that will not at the same time disturb the electrons enough to destroy the interference pattern.
That is, \textbf{Heisenberg's uncertainty principle}. Hence, in quantum mechanic, we can only make predictions of probability.

Remark that,

1. The evolution of quantum states is \textbf{definite} (either by Schr\"odinger equation or something else).

2. The uncertainty only appears in observations.

3. Quantum mechanic is an \textbf{extremely accurate} theory.