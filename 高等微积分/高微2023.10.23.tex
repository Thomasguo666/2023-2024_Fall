% !TeX root = 高等微积分.tex

复合极限定理中的修正二给出了$\lim_{x \to x_0} g\left( x \right) = g\left( x_0 \right) $, 这可以给出一个定义.

\begin{definition}
    设$f$在$x_0$的某开球邻域中有定义, 称$f$在$x_0$处连续, 如果以下条件之一成立:
    \begin{itemize}
        \item $\displaystyle \lim_{x \to x_0} f\left( x \right) = f\left( x_0 \right) $
        
        \item 对于$\forall \varepsilon > 0$, $\exists \delta > 0$, 当$\left| x - x_0 \right| < \delta$时, 有$\left| f\left( x \right) - f\left( x_0 \right)  \right| < \varepsilon$.
        
        \item 对与$f\left( x_0 \right) $的任何开球邻域$B_{\varepsilon}\left( f\left( x_0 \right)  \right) $, 都存在$x_0$的开球邻域, 使得
        \begin{equation}
            f\left( B_{\delta} \left( x_0 \right) \subseteq B_{\varepsilon}\left( f\left( x_0 \right)  \right)  \right).
        \end{equation}

        \item 对于$f \left( x_0 \right) $的任何一个邻域$V$, 都存在$x_0$的一个邻域$U$, 使得$f \left( U \right) \subseteq V$.
    \end{itemize}
\end{definition}
对于上面的定义, 我们可以通过不连续的例子来理解, $f$在$x_0$处不连续$\iff$

$\exists \varepsilon > 0$, $\forall \delta > 0$, $\exists \left| x - x_0 \right| < \delta $, 使得$\left| f\left( x \right) - f\left( x_0 \right)  \right| \geq \varepsilon$. 这可以说成$f$在$x_0$处撕开了定义域$D$.

\begin{example}
    判断连续性
    \begin{equation}
      f \left( x \right)  = \begin{cases} 
        \sin \frac{1}{x^{\alpha}}, & x \neq 0 
        \\ 
        0, & x = 0
      \end{cases}
    \end{equation}
    \textbf{解}

    当$\alpha > 0$时, $\lim_{x \to 0} f\left( x \right) $不存在, 故不连续.

    当$\alpha = 0$时, $\lim_{x \to 0} f \left( x \right) = \sin 1 \neq f \left( 0 \right) $. 故$f$在$0$处不连续.

    当$\alpha < 0$时, 令$\alpha = - \beta, \ (\beta > 0 )$, 则$\lim_{x \to 0} f \left( x \right) = 0 = f\left( 0 \right) $, 故$f$在$0$处连续.
\end{example}

\begin{definition}
    称$x$为$f$的连续点, 如果$f$在$x$处连续, 称$x$为$f$的间断点, 如果$f$在$x$处不连续.

    间断点也可以分为几类: 
    \begin{itemize}
        \item 本性间断点, $\displaystyle \lim_{x \to x_0} f\left( x \right) $不存在.
        \item 可去间断点, $\displaystyle \lim_{x \to x_0} f\left( x \right) $存在, 但不等于$f \left( x_0 \right) $
    \end{itemize}
\end{definition}
对于可去间断点, 我们可以通过定义$\tilde{f} \left( x_0 \right) = \begin{cases} 
  f \left( x \right) , & x \neq x_0 
  \\ 
  \lim_{x \to x_0} f\left( x \right), & x = x_0 
\end{cases} $使得$x_0$变为$\tilde{f}$的连续点.

\begin{proposition}[用序列极限刻画函数连续]
    $f\left( x \right) $ 连续当且仅当
    对于所有以 $x_0$ 为极限的点列 $\{ x_n \}_{n=1}^{\infty}$, 总有 $\displaystyle \lim_{n \to \infty} f\left( x_n \right) = f \left( x_0 \right) $.
\end{proposition}

\begin{proof}
    从充分性和必要性分别证明.

    \textbf{``$\implies$'':}
    
    设$f$在$x_0$处连续, 设$\displaystyle \lim_{n \to \infty} x_n = x_0$, 来证$\displaystyle \lim_{n \to \infty} f \left( x_n \right) = f \left( x_0 \right) $.

    这可以用复合极限定理来证明, 令$h \left( n \right) \equiv  x_n,\ \forall n \in \mathbb{Z}_{+}$, 则$\displaystyle \lim_{n \to \infty} h \left( n \right) = x_0$, 由于$f$在$x_0$处连续, 故
    \begin{equation}
        \lim_{n \to \infty} f \left( h \left( n \right) \right) = f \left( x_0 \right),
    \end{equation}
    这满足复合极限定理的修正二, 所以$\displaystyle \lim_{n \to \infty} f \left( x_n \right) = f \left( x_0 \right) $.

    \textbf{``$\impliedby$'':}

    设序列极限等于$f\left( x_0 \right) $成立, 来证$f$在$x_0$处连续, 即证$\displaystyle \lim_{x \to x_0} f\left( x \right)  = f\left( x_0 \right) $.

    \textbf{反证法}. 设$x\to x_0$时, $f\left( x \right) $不以$f\left( x_0 \right) $为极限, 即$\exists  \varepsilon > 0$, $\forall \delta > 0$, $\exists 0 < \left| x-x_0 \right| < \delta$, 使$\left| f\left( x \right) - f\left( x_0 \right) \right| \ge  \varepsilon$成立.
    
    特别的, 对$\delta = \frac{1}{n}$, 则$\exists 0 < \left| x_n - x_0 \right| < \frac{1}{n}$, 使得$\left| f\left( x_n \right) - f\left( x_0 \right)  \right| \geq \varepsilon$. 
    
    由于
    \begin{equation}
        \lim_{n \to \infty} f\left( x_n \right) = f\left( x_0 \right),
    \end{equation}
    矛盾!
\end{proof}

\begin{definition}
    设$f\colon D \to \mathbb{R}$, $D$满足$D$中每一点都有一个开球邻域包含在$D$中, 称$f$是$D$上的连续函数/映射, 记为$f \in C \left( D ; \mathbb{R} \right) $, 如果$f$在$D$的每一点处都连续.
\end{definition}

\begin{definition}
    称$D$是$\mathbb{R}$的一个开集(open set), 如果$\forall x_0 \in D$, 都存在$B_{r}\left( x_0 \right) \subseteq D$.
\end{definition}

下面我们可以考虑如何定义一般的映射的连续性.
\begin{definition}
    对于一般的$X, Y$, 称$f\colon X \to Y$在$x_0$处连续, 如果对于$Y$中任意开集$V$, $f^{-1} \left( V \right) $是$X$中的开集.
\end{definition}

\begin{definition}
    设$X$是一个集合, 所谓$X$上的一个拓扑结构, 是指$X$的一个子集族$\mathscr{T}$, 称$\mathscr{T}$的成员此拓扑的开集, 满足以下三条公理
    \begin{enumerate}
        \item $\varnothing, X \in \mathscr{T}$.
        \item $\mathscr{T}$中有限多个集合之交仍属于$\mathscr{T}$.
        \item $\mathscr{T}$中任意多个(可以是无穷个)集合之并仍属于$\mathscr{T}$.
    \end{enumerate}
    称$\left( X , \mathscr{T}\right) $为一个拓扑空间. 在上下文可以得出$\mathscr{T}$的时候, 也简称$X$为一个拓扑空间.
\end{definition}

\begin{example}
    $\mathscr{T}_{\text{平凡}} \equiv \{ \varnothing, X \}$称为平凡拓扑.
    $\mathscr{T}_{\text{离散}} \equiv \{ X \text{ 的所有子集} \} $称为离散拓扑.
\end{example}

在度量空间中, 我们可以有一些非平凡的拓扑. 令$\left( X, d \right) $为一个度量空间, 定义开球为$B_{r}\left( x \right)  = \{ y \in X | d\left( y,x \right) < r \}$ .令
\begin{equation}
    \mathscr{T} = \{ U \subseteq X | U \text{可以表示为开球之并} \}
\end{equation}
我们断言, $\mathscr{T}_{d}$满足拓扑公理, 称之为度量$d$诱导的拓扑.
\begin{proof}
    公理三是显然的. 公理一也是显然的, 构造如下
    \begin{equation}
      \varnothing = \text{0个开球之并},\quad X = \bigcup_{x \in X} B_{1}\left( x \right)
    \end{equation}

    公理二需要证明.

    设$U,V \in \mathscr{T}_{d}$, $U = \bigcup_{\alpha} U_{\alpha}$, $V = \bigcup_{\beta} V_{\beta}$. 
    \begin{equation}
        U \cap V = \bigcup_{\alpha, \beta} U_{\alpha} \cap V_{\beta} 
    \end{equation}
    我们只需证$U_{\alpha} \cap V_{\beta}$是开球之并即可. 
    设$U_{\alpha} = B_{r}\left( x \right)$, $V_{\beta} = B_{s}\left( y \right) $, 则对于
    \begin{equation}
      \forall z \in B_{r}\left( x \right) \cap B_{s}\left( y \right) 
    \end{equation}
    取
    \begin{equation}
      0 < t < \min \{ r - d \left( x,z \right) , s - d \left( y, z \right) \}.
    \end{equation}

    从而
    \begin{equation}
      B_{t} \left( z \right) \subseteq B_{r}\left( x \right) , \quad B_{t} \left( z \right) \subseteq B_{s}\left( y \right)
    \end{equation}
    所以有
    \begin{equation}
      B_{t}\left( z \right) \subseteq U_{\alpha} \cap V_{\beta}
    \end{equation}

    于是
    \begin{equation}
      U_{\alpha} \cap V_{\beta} = \bigcup_{z} B_{t} \left( z \right).
    \end{equation}
    我们就证明了开集之交仍是开集.
\end{proof}